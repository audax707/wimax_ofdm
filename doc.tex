\documentclass[10pt]{article}
\usepackage[utf8]{inputenc}
\usepackage{amsmath}
\usepackage{hyperref}
\usepackage[all]{hypcap}
\title{802.16-2009 OFDM PHY baseband}
\author{Jonathan Huang \and Cody Schafer \and Luke Steepy}
\date{2/3/2011}
\begin{document}
\maketitle
\section{Project Overview}
	\subsection{End Market Expectations}
	Currently IEEE 802.16, also known as “WiMAX”, has seen little deployment in the
	United States and western Europe while being moderately deployed in Asia
	nations and Eastern European markets. To be useful in western areas, WiMAX has
	to effectively compete with both consumer owned Wifi hot-spots and the various
	cell phone networks. Given the impressively low cost of both cellular and Wifi
	(802.11) systems combined with the expectation that they work well in mobile
	devices, this physical layer (PHY) targets both simplicity (to reduce cost) and
	reduced power consumption.

	\subsection{Degree of standards compliance and scope limitations}
	The 802.16 standard specifies multiple layers of a WiMAX device, including a
	“Service-specific convergence sublayer” (Section 5), a MAC sublayer (Section
	6), a Security sublayer (Section 7), and the Physical layer (Section 8). While
	the PHY is described within section 8 of the 802.16-2009 document, other
	sections may apply to the design of the PHY and thus were considered to the
	minimal extent to which they apply to design of the PHY component.  Within the
	PHY layer implementers are further given the choice of three (3) different PHY
	types: SC (single carrier), OFDM (orthogonal frequency division multiplexing),
	and OFDMA (orthogonal frequency division multiple access). In the interest of
	developing a product which is applicable to the larger consumer market and
	follows with our design goal of simplicity, OFDM was chosen as the modulation
	method to be implemented as it is significantly simpler in design than OFDMA
	and is designed for use on unlicensed and licenced bands. OFDM is detailed in
	section 8.3 of [1]

	FFT size fixed at 256

	% Need to figure out how to include images in Latex
	%\newpage
	%\section{Block Diagrams}
	%\subsection{Transmitter}
	% Sample includegraphics implementation
	%\begin{figure}[here]
	%\includegraphics[width=0.9\textwidth]{images/JobInformationDialog.jp%g}
	%\caption{A prototype of the Job Information dialog}
	%\label{fig:jobInfoDialog}
	%\end{figure}
	%\subsection{Receiver}

\section{Hardware Description of WiMAX PHY}
All voltages of logic levels are defined by the FPGA on which the product's
code is placed.

Each block of the transmitter connected via direct wiring (without a buffer)
is given the same clock. Except where explicitly mentiond, reading of an input
bit stream is done on the positive clock edge only when the corresponding
$load\_*$ control line is high. All PHY componets are clocked via the
$phy\_clock$. All wires noted as $enable$s (also: '$en$') are active high, and
remain high for the duration where the input stream is valid. Others indicated
as $flag$s are set on the first clock where they are valid, and will be
cleared on the next unless reasserted by the sending hardware.

	\subsection{MAC - PHY interface}
	\label{sec:mac_phy}
	The MAC interfaces with the PHY module by sending frames (with the
	appropriate headers and padding included) as a stream of bits. These
	bits are clocked via the $mac\_clk\_in$ line, and must only be sent
	when the $mac\_sending\_frame$ line is high. The state of the
	$mac\_sending\_frame$ line must be low when not sending a frame, and
	must be lowered and raised between frames that would otherwise be
	abutting. The frequency of $mac\_clk\_in$ must be less than or equal
	to half of the $phy\_clock$.

	The $phy\_ready$ line is a signal to the mac that it is ready for a
	new frame to be inputed, and should not be ignored.

	\subsection{Header Decoder}
	\label{sec:header}
	The Randomizer [\hyperref{sec:rand}] requires the locations of the
	PDUs (packet data units, 2 per frame) within each frame as well as the
	locations of 'Bursts' within each PDU to reinitialize itself at
	various points in the bit stream. As all of this data is contained in
	the PDU headers, this block determines the values of the needed fields
	and forwards them to the Control Logic block [\hyperref{sec:ctrl}]. 

	This block has the unique chalenge of additionally reading it's
	bitstream at the clock specified by the MAC, $mac\_bit\_clk\_in$.

	Inptus:
	Bit stream in: $bits\_to\_phy$, clocked via $mac\_bit\_clk\_in$.  No enable
	used. Reset via $header\_reset$.
	
	Outputs
	Bit stream out: $bits\_to\_rand$, Bit out enable: $en\_to\_rand$

	\subsection{Randomizer}
	\label{sec:rand}
	The randomizer operates as a shift register with an initial value
	determined by the PDU (packet data unit, one half of a frame) header
	and whether it is a UL (uplink) or DL (downlink) PDU.

	Inputs:
	Bit stream: $bits\_to\_rand$, bits input enable: $en\_to\_rand$.
	$burst\_change\_flag$ (indicates a reload of the register is needed,
	only high for a single bit).

	Outputs:
	Bit stream: $bits\_to\_fec$, bits output enable: $en\_to\_fec$.

	\subsection{Forward Error Corection}
	\label{sec:fec}
	This is a Reed-solomon and convolution coding combination which is
	applied per frame. Within the standard, different RS (Reed-Solomon)
	codes and CC (convolution code) rates are used for varying modulation
	types. As we have fixed the modulation to QPSK, only two of the RS
	code and CC rate pairs are needed, as indicated in the \autoref{tbl:fec}.

	\begin{tabular}{|c|c|c|c|c|c|}
	\label{tbl:fec}
		Modulation & Uncoded block size (bytes) & Coded block size (bytes) &
		Overall coding rate & RS code & CC code rate \\

		QPSK & 24 & 48 & 1/2 & (32,24,4) & 2/3 \\
		QPSK & 36 & 48 & 3/4 & (40,36,2) & 5/6
	\end{tabular}

	Bit stream in: $bits\_to\_fec$ (from Header decoder [\hyperref{sec:header}]),
	enable in: $en\_to\_fec$

	Bit stream out: $bits\_from\_fec$, $en\_from\_fec$.

	\subsection{FEC to Interleaving Circular buffer}
	\label{sec:fec_buffer}

	\subsection{Interleaving}
	\label{sec:interleaving}

	\subsection{Constellation Mapping}
	\label{sec:constellation}

	\subsection{Pilot Subcarrier Insertion}
	\label{sec:pilot}

	\subsection{Subcarrier to IFFT Buffer}
	\label{sec:ifft_in_buffer}

	\subsection{Cyclic Prefix}
	\label{sec:cyclic_prefix}

	\subsection{Control Logic}
	\label{sec:ctrl}

\section{Receiver PHY}
%The receiver portion of this device 
%
%\subsection{Pin outs}
%XC3S200-FT256
%//Top and Bottom View Package FT256 -   %http://www.xilinx.com/support/documentation/package_specs/ft256.pdf 
% \newpage
% \section{Datasheet(Spartan-3 FPGA}
% \subsection{Electrical Characteristics}
% \begin{tabular}
% \end{tabular}
\end{document}
