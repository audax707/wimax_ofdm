\documentclass[10pt]{article}
\usepackage[utf8]{inputenc}
\usepackage{amsmath}
\usepackage{hyperref}
\usepackage[english]{babel}
\usepackage{fullpage}
\usepackage[all]{hypcap}
\title{802.16-2009 OFDM PHY baseband}
\author{Jonathan Huang \and Cody Schafer \and Luke Steepy}
\date{2/14/2011}
\begin{document}
\maketitle
\section{Project Overview}
We are developing a minimal 802.16-2009 OFDM PHY supporting only QPSK
modulation and only components relevant to unlicensed bands. All
optional items are omitted.

	\subsection{End Market Expectations}
	Currently IEEE 802.16, also known as ``WiMAX'', has seen little deployment in the
	United States and western Europe while being moderately deployed in Asian
	nations and Eastern European markets. To be useful in western areas, WiMAX has
	to effectively compete with both consumer owned Wifi hot-spots and the various
	cell phone networks. Given the impressively low cost of both cellular and Wifi
	(802.11) systems combined with the expectation that they work well in mobile
	devices, this physical layer (PHY) targets both simplicity (to reduce cost) and
	reduced power consumption.

	\subsection{Degree of standards compliance and scope limitations}
	The 802.16 standard specifies multiple layers of a WiMAX device, including a
	``Service-specific convergence sublayer'' (Section 5), a MAC sublayer (Section
	6), a Security sublayer (Section 7), and the Physical layer (Section 8). While
	the PHY is described within section 8 of the 802.16-2009 document, other
	sections may apply to the design of the PHY and thus were considered to the
	minimal extent to which they apply to design of the PHY component.  Within the
	PHY layer, implementers are further given the choice of three (3) different PHY
	types: SC (single carrier), OFDM (orthogonal frequency division multiplexing),
	and OFDMA (orthogonal frequency division multiple access). In the interest of
	developing a product which is applicable to the larger consumer market and
	follows with our design goal of simplicity, OFDM was chosen as the modulation
	method to be implemented as it is significantly simpler in design than OFDMA
	and is designed for use on unlicensed and licensed bands. OFDM is detailed in
	section 8.3 of the 802.16-2009 document.

\section{Transmitter}
All voltages of logic levels are defined by the FPGA on which the product's
code is placed.

Each block of the transmitter connected via direct wiring (without a buffer) is
given the same clock. Each block reads its inputs on alternating clock edges
such that two adjacent units read and write on different edges. This is done so
that outputted data does not change while being read. Except where explicitly
mentioned, reading of an input bit stream is done on the positive clock edge
only when the corresponding $load\_*$ control line is high. All PHY components
are clocked via the $phy\_clock$. All wires noted as $enable$s (also: '$en$')
are active high, and remain high for the duration where the input stream is
valid. Others indicated as $flag$s are set on the first clock where they are
valid, and will be cleared on the next unless reasserted by the sending
hardware.

	\subsection{MAC - PHY interface}
	\label{sec:mac_phy}
	The MAC interfaces with the PHY module by sending frames (with the
	appropriate headers and padding included) as a stream of bits. These
	bits are clocked via the $mac\_clk\_in$ line, and must only be sent
	when the $mac\_sending\_frame$ line is high. The state of the
	$mac\_sending\_frame$ line must be low when not sending a frame, and
	must be lowered and raised between frames that would otherwise be
	abutting. The frequency of $mac\_clk\_in$ must be less than or equal
	to half of the $phy\_clock$.

	The $phy\_ready$ line is a signal to the mac that it is ready for a
	new frame to be inputed, and should not be ignored.

	\subsection{Header Decoder}
	\label{sec:header}
	The Randomizer (\autoref{sec:rand}) requires the locations of the
	PDUs (packet data units, 2 per frame) within each frame as well as the
	locations of 'Bursts' within each PDU to reinitialize itself at
	various points in the bit stream. As all of this data is contained in
	the PDU headers, this block determines the values of the needed fields
	and forwards them to the Control Logic block (\autoref{sec:ctrl}). 

	This block has the unique challenge of additionally reading it's
	bit-stream at the clock specified by the MAC, $mac\_bit\_clk\_in$.

	Inputs: \\
	Bit stream in: $bits\_to\_phy$, clocked via $mac\_bit\_clk\_in$.  No
	enable used. Uses $mac\_frame\_flag$ to determine when a new frame is
	inbound.
	
	Outputs: \\
	Bit stream out: $bits\_to\_rand$, Bit out enable: $en\_to\_rand$

	Estimated gates: 1k \\
	Responsible person: Cody

	\subsection{Randomizer}
	\label{sec:rand}
	The randomizer operates as a shift register with an initial value
	determined by the PDU (packet data unit, one half of a frame) header
	and whether it is a UL (uplink) or DL (downlink) PDU.

	Inputs: \\
	Bit stream: $bits\_to\_rand$, bits input enable: $en\_to\_rand$.
	$burst\_change\_flag$ (indicates a reload of the register is needed,
	only high for a single bit).

	Outputs: \\
	Bit stream: $bits\_to\_fec$, bits output enable: $en\_to\_fec$.

	Estimated Gates: 1k \\
	Responsible Person: Cody.

	\subsection{Forward Error Correction}
	\label{sec:fec}
	This is a Reed-Solomon and convolution coding combination which is
	applied per frame. Within the standard, different RS (Reed-Solomon)
	codes and CC (convolution code) rates are used for varying modulation
	types. As we have fixed the modulation to QPSK, only two of the RS
	code and CC rate pairs are needed, as indicated in the \autoref{tbl:fec}.
	
	

	%\begin{table}
		\begin{tabular}{p{2cm}|p{2cm}|p{2cm}|p{2cm}|p{2cm}|p{2cm}}
		\label{tbl:fec}
			Modulation & Uncoded block size (bytes) & Coded block size (bytes) &
			Overall coding rate & RS code & CC code rate \\ \hline
			QPSK & 24 & 48 & 1/2 & (32,24,4) & 2/3 \\
			QPSK & 36 & 48 & 3/4 & (40,36,2) & 5/6
		\end{tabular}
	%	\caption{}
	%\end{table}


	Bit stream in: $bits\_to\_fec$ (from Header decoder, \autoref{sec:header}),
	enable in: $en\_to\_fec$

	Bit stream out: $bits\_from\_fec$, $en\_from\_fec$.

	Estimated Gates: 2k \\
	Responsible Person: Luke.

	\subsection{FEC to Interleaving Circular buffer}
	\label{sec:fec_buffer}

	Takes $bits\_from\_fec$ (triggered via $en\_from\_fec$) and places them
	in a circular buffer. When the number of buffered bits equals the block
	size of the Interleaver. Sets the $fec\_buffer\_done$ flag when enough
	data is filled. Data is outputted via $fec\_buffer\_out$ with a width
	equal to the block size of the interleaver.

	Estimated Gates: 1K \\
	Responsible Person: Cody

	\subsection{Interleaving}
	\label{sec:interleaving}

	Two step type operation, swaps bits around.

	Estimated Gates: 2K \\
	Responsible Person: Cody

	\subsection{Constellation Mapping}
	\label{sec:constellation}

	Input is bit stream, output is I \& Q pair.

	Estimated Gates: 4K \\
	Responsible Person: Cody

	\subsection{Pilot Subcarrier Insertion}
	\label{sec:pilot}

	Input is I \& Q pair, output is I \& Q pair with pilot subcarriers
	inserted at some points (fixed).

	Estimated Gates: 1K \\
	Responsible Person: Cody

	\subsection{Subcarrier to IFFT Buffer}
	\label{sec:ifft-buffer}
	Insertion into this buffer is ordered but will have random jumps around
	different UL \& DL bursts.

	Block Size = blk\_siz = 200 (the total number of used subcarriers)
	
	Buffers blk\_size items each of which has
	width 2 bits. QPSK utilizes both I and Q, amplitude and phase, each with a
	granularity of 2. This means that each item (a pair of I \& Q) has 4
	possible values and thus 2 bits are needed for each.

	Estimated Gates: 2k
	Responsible Person: Cody.

	\subsection{IFFT}
	\label{sec:ifft}
	IFFT Size: 256 items (each being a 
	Inputs: 
		$ifft\_buf\_has\_block$ (the count of the last block
		present in the Sub to IFFT buffer (\autoref{sec:ifft-buffer})
		
		$ifft\_buf\_data[200][2]$ (access to the data at the present 'tail' location
		in the buffer
	
	Outputs:
		$ifft\_data\_ready$ high when the IFFT has processed data.

	Estimated Gates: 10k.
	Responsible Person: John.

	\subsection{Cyclic Prefix}
	\label{sec:cyclic_prefix}

	Inputs:
		$ifft\_done$ : indicates the IFFT has data to be read out.
		$ifft\_data$ : single bit of data from the IFFT.

	Output:
		$ifft\_shift\_clk$: clocks out the data from the IFFT.
		$bits\_cyclic\_appended[2]$: sequence of I\&Q with the prefix appended.
			This is sent directly to the ADC.


	Appends the last Tg items of the previous frame to the start of the
	present frame.  Reads 256 2 bit items from the IFFT, appending Tg items
	to the start of it. It also stores the last Tg items so that they may
	be appended to the next frame from the IFFT.

	Estimated Gates: 2k.
	Responsible Person: Cody.

	\subsection{Control Logic}
	\label{sec:ctrl}

\section{Receiver PHY}
	\subsection{Pilot Locater}

	\subsection{Pilot locater to FFT buffer}

	\subsection{FFT}

	\subsection{Constellation De-mapping}
	
\section{FPGA Implementation}
  \subsection{Sampling and Clock Rate}
    From the IEEE 802.16 standards document, the sampling rate for 256-OFDM is defined as
    \begin{equation}
    F_s = BW * 7/6
    \end{equation}
  
  \begin{center}
  \begin{tabular}{|c|c|}
  Channel Bandwidth & Sampling Rate \\ \hline
  14 MHz & 16.333 MHz \\
  7 MHz & 8.166 MHz \\
  3.5 MHz & 4.083 MHz \\
  1.75 MHz & 2.042 MHz \\ \hline
  \end{tabular}
  \end{center}
  
  This design should be able to run on both the Spartan-3 FPGA and Virtex 4 LX60
  FPGA as they both supply a 500 MHz clock.
  
  \subsection{Power Calculation}
  
  Calculate power based on required clock frequency for the device as well as the
  FPGA specifications for voltage inputs.
  
  Absolute Maximum rating for V_{in} : 4.4 V
  
  Clock frequency required for a 3.5 MHz channel bandwidth : 4.083 MHz
  
  Input capacitance over recommended operating conditions : 10 pF
  \begin{center}
  \begin{equation}
  P =C * V^2 * f = 10 pF * (4.4 V)^2 * 4.083 MHz = 0.18 mW
  \end{equation}
  \end{center}
  
	%\subsection{Pin outs}
	%XC3S200-FT256
	%//Top and Bottom View Package FT256 -   %http://www.xilinx.com/support/documentation/package_specs/ft256.pdf 
	%\section{Datasheet(Spartan-3 FPGA}
	%\subsection{Electrical Characteristics}
	%\begin{tabular}
	%\end{tabular}
\end{document}
