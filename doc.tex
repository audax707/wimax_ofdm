\documentclass[10pt]{article}
\usepackage[utf8]{inputenc}
\usepackage{amsmath}
\title{802.16-2009 OFDM PHY baseband}
\author{Jonathan Huang \and Cody Schafer \and Luke Steepy}
\date{2/3/2011}
\begin{document}
\maketitle
\section{Project Overview}
	\subsection{End Market Expectations}
	Currently IEEE 802.16, also known as “WiMAX”, has seen little deployment in the
	United States and western Europe while being moderately deployed in Asia
	nations and Eastern European markets. To be useful in western areas, WiMAX has
	to effectively compete with both consumer owned Wifi hot-spots and the various
	cell phone networks. Given the impressively low cost of both cellular and Wifi
	(802.11) systems combined with the expectation that they work well in mobile
	devices, this physical layer (PHY) targets both simplicity (to reduce cost) and
	reduced power consumption.

	\subsection{Degree of standards compliance and scope limitations}
	The 802.16 standard specifies multiple layers of a WiMAX device, including a
	“Service-specific convergence sublayer” (Section 5), a MAC sublayer (Section
	6), a Security sublayer (Section 7), and the Physical layer (Section 8). While
	the PHY is described within section 8 of the 802.16-2009 document, other
	sections may apply to the design of the PHY and thus were considered to the
	minimal extent to which they apply to design of the PHY component.  Within the
	PHY layer implementers are further given the choice of three (3) different PHY
	types: SC (single carrier), OFDM (orthogonal frequency division multiplexing),
	and OFDMA (orthogonal frequency division multiple access). In the interest of
	developing a product which is applicable to the larger consumer market and
	follows with our design goal of simplicity, OFDM was chosen as the modulation
	method to be implemented as it is significantly simpler in design than OFDMA
	and is designed for use on unlicensed and licenced bands. OFDM is detailed in
	section 8.3 of [1]

	FFT size fixed at 256

	% Need to figure out how to include images in Latex
	%\newpage
	%\section{Block Diagrams}
	%\subsection{Transmitter}
	% Sample includegraphics implementation
	%\begin{figure}[here]
	%\includegraphics[width=0.9\textwidth]{images/JobInformationDialog.jp%g}
	%\caption{A prototype of the Job Information dialog}
	%\label{fig:jobInfoDialog}
	%\end{figure}
	%\subsection{Receiver}

\section{Transmitter PHY}
All voltages of logic levels are defined by the FPGA on which the product's
code is placed.

Each block of the transmitter connected via direct wiring (without a buffer)
is given the same clock. Except where explicitly mentiond, reading of an input
bit stream is done on the positive clock edge only when the corresponding
$load\_*$ control line is high. All PHY componets are clocked via the
$phy\_clock$.

	\subsection{MAC - PHY interface}
	The MAC interfaces with the PHY module by sending frames (with the
	appropriate headers and padding included) as a stream of bits. These
	bits are clocked via the $mac\_clk\_in$ line, and must only be sent
	when the $mac\_sending\_frame$ line is high. The state of the
	$mac\_sending\_frame$ line must be low when not sending a frame, and
	must be lowered and raised between frames that would otherwise be
	abutting. The frequency of $mac\_clk\_in$ must be less than
	half of the $phy\_clock$ unless both clocks share their phase, in
	which case $mac\_clk\_in$ may be equal to the full $phy\_clock$.

	The $phy\_ready$ line is a signal to the mac that it is ready for a
	new frame to be inputed, and should not be ignored.

	\subsection{Header Decoder}
	The \cite{Randomizer} requires the locations of the PDUs (packet data
	units, 2 per frame) within each frame as well as the locations of
	'Bursts' within each PDU to reinitialize itself at various points in
	the bit stream. As all of this data is contained in the PDU headers,
	this block determines the values of the needed fields and forwards
	them to the \cite{Control Logic} block.

	This block has the unique chalenge of additionally reading it's
	bitstream at the clock specified by the MAC, $mac\_clk\_in$.

	Bit stream in: $mac\_bits$, clocked via $mac\_clk\_in$.
	No enable used. Reset via $header\_reset$. Bit stream out:
	$bits\_norm$ (indicating their clocking is now normalized).

	\subsection{Randomizer}
	The randomizer operates as a shift register with an initial value
	determined by the PDU (packet data unit, one half of a frame) header
	and whether it is a UL (uplink) or DL (downlink) PDU. 

%
%\begin{tabular}{|c|c|c|}
%\hline
%Block Name & Description & Required Inputs \\ \hline
%Randomizer & Each frame of data is put through a Randomizer which utilizes a PRBS \newline
% 
%(Psuedo Random Binary Sequence) generator \newline with $1+x^{14}+x^{15}$ & From Header Parser : Contents of the frame \newline
%From Control Logic : Randomizer Initialization Vector and Reset Control \\ \hline
%FEC & Forward Error Correction consisting of Concatenated Reed Solomon-Convolution Code & From Randomizer : New random bit stream and Sending Data signal \newline
% 
%From FEC Calculation : RS Code and CC Code Rate \\ \hline
%Circular Buffer & Holds blocks of encoded bits for the Interleaver & From FEC : Encoded bits and Sending Data signal \\ \hline
%Interleaver & Interleaves data blocks & From Circular Buffer : Blocks of data \\ \hline
%Constellation Mapping & Maps bits according to the modulation type & From Interleaver : Interleaved Bits and Sending Data signal \newline
% 
%From Control Logic : Subchannelization information \\ \hline
%Pilot Subcarrier insertion & & From Constellation Mapping : Subcarrier mappings (I and Q) \\ \hline
%IFFT & Inverse Fast Fourier Transform & From Pilo Subcarrier insertion : Subcarrier mappings with Pilots \\ \hline
% 
%Cyclic Prefix & Appends the Cyclic Prefix to the signal & From IFFT : I and Q Transform 
% 
%From Control Logic : Cyclic Prefix size
% 
%DAC Converter / Wave Modulation & & \\ \hline
%\end{tabular}
\subsubsection{Receiver MAC/PHY}
%The receiver portion of this device 
%
%\subsection{Pin outs}
%XC3S200-FT256
%//Top and Bottom View Package FT256 -   %http://www.xilinx.com/support/documentation/package_specs/ft256.pdf 
% \newpage
% \section{Datasheet(Spartan-3 FPGA}
% \subsection{Electrical Characteristics}
% \begin{tabular}
% \end{tabular}
\end{document}
